\documentclass{standalone}

\usepackage{lmodern}
\usepackage{tikz}
\usetikzlibrary{positioning, chains, calc, arrows.meta, backgrounds}

% define layers
    \pgfdeclarelayer{foreground}
    \pgfdeclarelayer{background}
% tell TikZ how to stack them (back to front)
    \pgfsetlayers{background,main,foreground}

\begin{document}

\tikzset{trans-node/.style = {draw, line width = 3pt, inner sep = 12pt}}
\tikzset{read-from/.style = {>={Stealth[length = 12pt]}, ->, dash pattern = on 20pt off 10pt, thick}}
\tikzset{knode/.style = {inner sep = 8pt, outer sep = 5pt, font=\fontsize{40}{40}\selectfont}}
\tikzset{toarrow/.style = {>={Stealth[length = 12pt]}, ->, ultra thick}}
\tikzset{dashline/.style = {ultra thick, dash pattern = on 20pt off 10pt}}
\tikzset{cs/.style = {outer sep = 5pt, font=\fontsize{40}{40}\selectfont}}  % c: commit; s: start

% #1: numbering
\newcommand{\xtrans}[1]{\node (x#1) [trans-node, on chain = x, font=\fontsize{30}{30}\selectfont] {$T_{x_#1}: w_{x_#1}(x_#1)$}}
\newcommand{\ytrans}[1]{\node (y#1) [trans-node, on chain = y, font=\fontsize{30}{30}\selectfont] {$T_{y_#1}: w_{y_#1}(y_#1)$}}

\begin{tikzpicture}[start chain = x going right,
  		start chain = y going right,
	      	node distance = 1.0cm,
	        font = \huge]
  % transactions updating x
  \begin{scope}
    \foreach \i in {1, ..., 6}
      \xtrans{\i};
  \end{scope}

  % transactions updating y
  \begin{scope}
    \node (y1) [trans-node, on chain = y, below left = 3.0cm and -2.0cm of x3, font=\fontsize{30}{30}\selectfont] {$T_{y_1}: w_{y_1}(y_1)$};
    \foreach \i in {2, ..., 6}
      \ytrans{\i};
  \end{scope}
   
  % transaction T_i reading x and y
  \begin{scope}
    \node (ti) [trans-node, below right = 2.0cm and 2.5cm of y2, font=\fontsize{35}{35}\selectfont] 
	  {$T_i: \hskip 3em r_i(x_2) \hskip 8em r_i(y_5) \hskip 4em$};
  \end{scope}

  % T_i reads x2
  \begin{pgfonlayer}{background}
	\coordinate (c-rx2) at ($(ti.north) !0.5! (ti.north west)$);
	\draw[read-from]  (x2.south) to (c-rx2); % [out = -70, in = 120] (c-rx2);
  \end{pgfonlayer}
  
  % T_i reads y5
  \coordinate (c-ry5) at ($(ti.north) !0.5! (ti.north east)$);
  \draw[read-from] (y5.south) to (c-ry5);

  % k1
  \def\konespace{11cm}
  \begin{scope}[on background layer]
      % c-x2-top: commit time of T_x2
      \draw let \p{p-x2} = (x2.east), \p{p-ti} = (ti.north west) in
	  node (c-x2-top) [cs] at ($(\x{p-x2}, \y{p-ti}) + (0.0cm, \konespace)$) {$c_{x_2}$};
      \draw[dashline] (x2.north east) to (c-x2-top.south);

      % s-ti-top: start time of T_i (drawn at the top)
      \node (s-ti-top) [cs] at ($(ti.north west) + (0.0cm, \konespace)$) {$s_i$};
      \draw[dashline] let \p{p-ti} = (ti.west), \p{p-x4} = (x4.west) in
		(s-ti-top.south) to (\x{p-ti}, \y{p-x4});

      % k1 at center of c-x2-top and s-ti-top
      \node (k1) [knode] at ($(c-x2-top) !0.5! (s-ti-top)$) {$k_1$};
      \draw [toarrow] (k1) to (c-x2-top);
      \draw [toarrow] (k1) to (s-ti-top);
  \end{scope}

  % k2
  \def\ktwospace{1.5cm}
  \begin{scope}[on background layer]
      % s-ti-bot: start time of T_i (drawn at the bottom)
      \node (s-ti-bot) [cs] at ($(ti.south west) - (0.0cm, \ktwospace)$) {$s_i$};
      \draw[dashline] let \p{p-ti} = (ti.west), \p{p-x4} = (x4.west) in
      (s-ti-bot.north) to (\x{p-ti}, \y{p-x4});

      % c-y5-bot: commit time of T_y5
      \draw let \p{p-y5} = (y5.east), \p{p-ti} = (ti.south west) in
      node (c-y5-bot) [cs] at ($(\x{p-y5}, \y{p-ti}) - (0.0cm, \ktwospace)$) {$c_{y_5}$};
      \draw [dashline] (y5.south east) to (c-y5-bot.north);

      % k2 at center of s-ti-bot and c-y5-bot
      \node (k2) [knode] at ($(s-ti-bot) !0.5! (c-y5-bot)$) {$k_2$};
      \draw [toarrow] (k2) to (s-ti-bot);
      \draw [toarrow] (k2) to (c-y5-bot);
  \end{scope}

  % k3
  \begin{scope}[on background layer]
      % (invisible; as positioning anchor) s-ti-mid: start time of T_i (drawn in the middle)
      \node (s-ti-mid) [cs] at ($(s-ti-top) !0.35! (s-ti-bot)$) {};

      % c-x2-bot: commit time of T_x2 (drawn at the bottom)
      \draw let \p{p-x2} = (x2.east), \p{p-s-ti-mid} = (s-ti-mid) in
        node (c-x2-bot) [cs] at (\x{p-x2}, \y{p-s-ti-mid}) {$c_{x_2}$};
      \draw[dashline] (x2.south east) to (c-x2-bot.north);

      % c-y5-top: commit time of T_y5 (drawn at the top)
      \draw let \p{p-y5} = (y5.east), \p{p-s-ti-mid} = (s-ti-mid) in
        node (c-y5-top) [cs] at (\x{p-y5}, \y{p-s-ti-mid}) {$c_{y_5}$};
      \draw[dashline] (y5.north east) to (c-y5-top.south);

      % k3 at center of c-x2-bot and c-y5-top
      \node (k3) [knode] at ($(c-x2-bot) !0.5! (c-y5-top)$) {$k_3$};
      \draw [toarrow] (k3) to (c-x2-bot);
      \draw [toarrow] (k3) to (c-y5-top);
  \end{scope}
\end{tikzpicture}

\end{document}
